\documentclass[11pt]{scrartcl}
\usepackage[a4paper]{geometry}

\usepackage{graphicx}
%\graphicspath{ {./images/} }

\usepackage{fancyhdr}
\pagestyle{fancy}
\fancyhf{}
\fancyhead[L]{ULTRASCHALL - DOPPLER} %Kopfzeile links
\fancyfoot[C]{\thepage}

\usepackage[utf8]{inputenc}
\usepackage{csquotes}
\usepackage[german]{babel}

\usepackage{setspace}

\usepackage{caption}
\usepackage{float}

\usepackage{hyperref}
\usepackage{pdfpages}

\hypersetup{
    pdftitle = {Ultraschall-Doppler},
    pdfsubject = {Biomedizinischesystemtechnik Praktikum},
    pdfauthor = {Leona K{\"o}ck, Chris R{\"u}ttimann},
    pdfkeywords = {} ,
    pdfcreator = {pdflatex},
    pdfproducer = {LaTeX with hyperref}
}

\usepackage[
    style=apa,
    backend=biber,
    sortcites=false,
    sorting=none,
    hyperref=true,
    backref=false
]{biblatex}
\usepackage{amsmath}
\usepackage[T1]{fontenc}
%\addbibresource{test.bib}

\begin{document}
    \pagenumbering{Alph}
% ---------------------
% Titlepage
% ---------------------
    \begin{titlepage}
        \begin{center}
        {\LARGE OST Ostschweizer Fachhochschule}
            \\[1.5cm]
            \linespread{1.2}\large { Biomedizinischesystemtechnik Praktikum }

            \huge{\bfseries Ultraschall-Doppler}
            \\%[1.5cm]
            \large{durchgef{\"u}hrt am 22. März 2021}
            \\[1.5cm]
   %         \linespread{1}
           \includegraphics[width=8cm]{../images/ost_logo.eps}
           \\[1cm]
            {\small{Autoren}}\\
            {\Large{Leona K{\"o}ck}}\\
            {\Large{Chris R{\"u}ttimann}}
            \\[1cm]

            \vspace*{\fill}
            \large{\today}
        \end{center}

    \end{titlepage}

% ---------------------
% Abstract
% ---------------------
    \pagenumbering{Roman}
 %   \pdfbookmark[section]{Abstract}{abstract}
 %   \section*{Abstract}
    \addtocounter{section}{0}

 %   \pagebreak
    \setstretch{1.25}
% ---------------------
% Table of contents
% ---------------------
    \tableofcontents
    \pagebreak


% ---------------------
% Body
% ---------------------
    \pagenumbering{arabic}

    \section{Problem- und Zielvorstellung}
    Ziel dieses Praktikums war es, die Vorteile der nichtinvasiven Messmethode nach dem Prinzip \emph{Continuous Wave Doppler}
    kennenzulernen und sowie die bereits vorhandenen Kenntnisse aus der Vorlesung mit praktischen Versuchen zu vertiefen. 
    \section{Problemlösung}
    \subsection{Vorbereitung}
   Das Praktikum wurden anhand der Angaben aus \cite{Doppler} durchgeführt.
    
    Für der Versuch wurden folgende Materialien benötigt:
    \begin{itemize}
        \item Dopplergerät HiDop 360 
        \item PC mit der Software HiDop
        \item 4MHz und 8MHz Transducer
        \item 4MHz Test-Transducer 
        \item Halterung für zwei Transducer
        \item Funktionsgenerator HMF 2550 
        \item Gel
    \end{itemize}


    \subsection{Messung}
    \subsubsection{HiDop 360}
    Die erste Aufgabe bestand darin, sich mit dem Dopler-Messgerät vertraut zu machen 
    und die wichtigsten Funktionen kennen zu lernen.
    Dazu gehörte unter anderem die zwei Sonden mit dem MEssgerät zu verbinden, das Messgerät 
    wiederum mit dem PC zu verbinden sowie das Programm HiDop zu starten.
    Es wurde ein Patient angelegt um die folgenden Messungen speichern zu können und somit auch die ersten Messungen durchgeführt. 
    %Eventuell Messaufbau zeichnen? 

    \subsubsection{Ausmessen des Dopplergerätes HiDop 360}
    Die zweite Aufgabe war es, mithilfe eines Sonogramms zu überprüfen, ob das Dopplergerät funktioniert und richtig geeicht ist. 
    Dazu wurde der der 4MHz Transducer des Messgeräts sowie der Testtranducer in die Halterung mit ca. einem Millimeter Abstand eingespannt.
    Um eine gute Übertragung des Signals zu gewährleisten ist der Zwischenraum mit Gel gefüllt worden. 
    Der Testtranducer war mithilfe eines Abschwächers an den Funktionsgenerator, der ein 4.001MHz Sinussignal liefert, angeschlossen. 
    Die Verbindung mit dem PC wurde genutzt um das Sonogramm besser darzustellen und speichern zu könnnen.
    Am Gerät selbst wurde der 5kHz Messbereich, eine Zeitablenkung von 4s sowie die Sonogrammdarstellung gewählt.

    \subsubsection{Testmessung an Gefässen}
    Um Messungen an den Gefässen der Probanden vorzunehmen wurde auf den 8MHz Transducer gewechselt. 
    Am Dopplergerät wurdend ie Doppler-indizies S/D und RI eingestellt. 
    Bei den beiden Probenden Chris Rüttimann und Leona Köck wurden Messungen sowohl an der Carotis Communis (Halsschalgader), als auch an der Arteria Carotis (Handgelenk) durchgeführt.

    \subsubsection{FFT}

    \section{Ergebnisse}
    \subsubsection{Ausmessen des Dopplergerätes HiDop 360}
    Wie nach lesen der Angabe zu erwarten war, stimmte die gemessene Frequenz des Dopplergeräts nicht genau mit der des Funtionsgenerators überein.
    Zu sehen ist dies in der Abbildung 
    Das ist der Fall, weil der Funktionsgenerator genauer ist als des medizinischen Messgeräts.
    Die Frequenz des Quarzes vond dem Dopplergerät weicht aufrund von Temperatur und Alter ab. 
    Diese Differenz wird in ppm angegeben, wobei 100ppm bei medizinischen Messgeräten dem Standard entspäricht.
    Berechnet man dies, würde das für diese Messung schon eine Differenz von 400Hz bedeuten. 
    Die Abweichung ist bei dem Messgerät nicht von großer Bedeutung, da durch den verwendeten Demodulator lediglich das Differenzsignal (=Dopplerfrequenz) erhalten bleibt.\\
    Um die angestrebten 4.001MHz zu erreichen wurden der das Signal des Funktionsgenerators um 180Hz erhöht. 

    \begin{figure}[H]
        \includegraphics[]{../images/4001MHz.png}
    \end{figure}

    \subsubsection{Testmessung an Gefässen}
     Frequenz nicht so wichtig .... Muster ist wichtiger ... ist aber bei jeder Stelle anderst und andere 
     Ausprägungen bedeuten andere Sachen ⇒ sehr komplex 
    Messwinkel ca. 60 Grad beachten (in Flussrichtung) 
    ... Bewegungsartefakte vermeiden
    messungen am liegenden Patient würden bessere ERgebnisse erzielen 
    % deutung: https://www.kup.at/kup/pdf/3947.pdf sieht vielversprechend aus... weiß aber nicht ob wir das brauchen, wird nicht erwartet ... 

     Chris: sehr leicht
     Leona: schwer zu finden.... auf Ton hören hilft 
    \begin{figure}[H]
        \includegraphics[]{../images/Chris_Hals.png}
    \end{figure}
    \begin{figure}[H]
        \includegraphics[]{../images/Chris_Handgelenk.png}
    \end{figure}
    \begin{figure}[H]
        \includegraphics[]{../images/Leona_Hals.png}
    \end{figure}
    \begin{figure}[H]
        \includegraphics[]{../images/Leona_Handgelenk.png}
    \end{figure}
    \subsubsection{FFT}

   
    \section{Kritik und Anregungen}
	% Was ned ob ma des brucht
    \pagebreak

    \section*{Eigenständigkeitserklärung}
    \addcontentsline{toc}{section}{Eigenständigkeitserklärung}

    Hiermit bestätigen wir, dass wir diesen Bericht selbstständig und ohne fremde Hilfe verfasst haben.
    Alle verwendeten Quellen wurden entsprechend dem APA-Standard gekennzeichnet.
    \\[3cm]


    \begin{figure}[H]
        \includegraphics[width=4cm]{.././images/Unterschrift_Leona.png}
    \end{figure}
    \begin{tabular}{@{} l@{}}
        \hline \\
        \makebox[6cm]{Leona Köck}\\[2cm]
    \end{tabular}


    \begin{figure}[H]
        \includegraphics[width=4cm]{.././images/Unterschrift_Chris.png}
    \end{figure}
    \begin{tabular}{@{} l@{}}
        \hline\\
        \makebox[6cm]{Chris Rüttimann}
    \end{tabular}

    \pagebreak
% ---------------------
% References
% ---------------------
    %\printbibliography
    \addcontentsline{toc}{section}{Literaturverzeichnis}

% ---------------------
% List of figures
% ---------------------
%    \listoffigures
%    \addcontentsline{toc}{section}{Abbildungsverzeichnis}
%    \pagebreak

% ---------------------
% List of tables
% ---------------------
%\listoftables



% ---------------------
% Anhang
% ---------------------
%\appendix

\end{document}